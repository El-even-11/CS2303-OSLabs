% !TEX root = ../main.tex

\chapter{Maxwell Equations}

选择二维情况,有如下的偏振矢量:
\begin{subequations}
  \begin{align}
    {\bf E} &= E_z(r, \theta) \hat{\bf z}, \\
    {\bf H} &= H_r(r, \theta) \hat{\bf r} + H_\theta(r, \theta) \hat{\bm\theta}.
  \end{align}
\end{subequations}
对上式求旋度:
\begin{subequations}
  \begin{align}
    \nabla \times {\bf E} &= \frac{1}{r} \frac{\partial E_z}{\partial\theta}
      \hat{\bf r} - \frac{\partial E_z}{\partial r} \hat{\bm\theta}, \\
    \nabla \times {\bf H} &= \left[\frac{1}{r} \frac{\partial}{\partial r}
      (r H_\theta) - \frac{1}{r} \frac{\partial H_r}{\partial\theta} \right]
      \hat{\bf z}.
  \end{align}
\end{subequations}
因为在柱坐标系下,$\overline{\overline\mu}$ 是对角的,所以 Maxwell 方程组中电场
$\bf E$ 的旋度:
\begin{subequations}
  \begin{align}
    & \nabla \times {\bf E} = \ii \omega {\bf B}, \\
    & \frac{1}{r} \frac{\partial E_z}{\partial\theta} \hat{\bf r} -
      \frac{\partial E_z}{\partial r}\hat{\bm\theta} = \ii \omega \mu_r H_r
      \hat{\bf r} + \ii \omega \mu_\theta H_\theta \hat{\bm\theta}.
  \end{align}
\end{subequations}
所以 $\bf H$ 的各个分量可以写为:
\begin{subequations}
  \begin{align}
    H_r &= \frac{1}{\ii \omega \mu_r} \frac{1}{r}
      \frac{\partial E_z}{\partial\theta}, \\
    H_\theta &= -\frac{1}{\ii \omega \mu_\theta}
      \frac{\partial E_z}{\partial r}.
  \end{align}
\end{subequations}
同样地,在柱坐标系下,$\overline{\overline\epsilon}$ 是对角的,所以 Maxwell 方程
组中磁场 $\bf H$ 的旋度:
\begin{subequations}
  \begin{align}
    & \nabla \times {\bf H} = -\ii \omega {\bf D}, \\
    & \left[\frac{1}{r} \frac{\partial}{\partial r}(r H_\theta) - \frac{1}{r}
      \frac{\partial H_r}{\partial\theta} \right] \hat{\bf z} = -\ii \omega
      {\overline{\overline\epsilon}} {\bf E} = -\ii \omega \epsilon_z E_z
      \hat{\bf z}, \\
    & \frac{1}{r} \frac{\partial}{\partial r}(r H_\theta) - \frac{1}{r}
      \frac{\partial H_r}{\partial\theta} = -\ii \omega \epsilon_z E_z.
  \end{align}
\end{subequations}
由此我们可以得到关于 $E_z$ 的波函数方程:
\begin{equation}
  \frac{1}{\mu_\theta \epsilon_z} \frac{1}{r} \frac{\partial}{\partial r}
  \left(r \frac{\partial E_z}{\partial r} \right) + \frac{1}{\mu_r \epsilon_z}
  \frac{1}{r^2} \frac{\partial^2E_z}{\partial\theta^2} +\omega^2 E_z = 0.
\end{equation}
