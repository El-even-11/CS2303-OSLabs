% !TeX encoding = UTF-8

% 这个模版是 SJTUTeX v2 的测试模版。如果您是 SJTUThesis 示例模版的用户,请无视这个文件。

% 载入 SJTUThesis 模版
\documentclass[type=bachelor]{sjtuthesis}

\title{上海交通大学学位论文 \LaTeX{} 模板示例文档}

\sjtusetup{
  info = {
    title           = {上海交通大学学位论文 \LaTeX{} 模板示例文档},
    title*          = {A Sample Document for \LaTeX-based SJTU Thesis Template},
    % display-title   = {上海交通大学学位论文\\ \LaTeX{} 模板示例文档},
    % display-title*  = {A Sample Document \\ for \LaTeX-based SJTU Thesis Template},
    % running-title   = {示例文档},
    % running-title*  = {Sample Document},
    keywords        = {上海交大, 饮水思源, 爱国荣校},
    keywords*       = {SJTU, master thesis, XeTeX/LaTeX template},
    author          = {某\quad{}某},
    author*         = {Mo Mo},
    supervisor      = {某某教授},
    supervisor*     = {Prof. Mou Mou},
    assoc-supervisor  = {某某教授},
    assoc-supervisor* = {Prof. Uom Uom},
    id              = {0010900990},
    degree          = {工学硕士},
    degree*         = {Master of Engineering},
    major           = {某某专业},
    major*          = {A Very Important Major},
    department      = {某某系},
    department*     = {Depart of XXX},
  },
  style = {
    header-logo-color = blue,
    title-logo-color = blue,
  },
}

% 使用 BibLaTeX 处理参考文献
%   biblatex-gb7714-2015 常用选项
%     gbnamefmt=lowercase     姓名大小写由输入信息确定
%     gbpub=false             禁用出版信息缺失处理
\usepackage[backend=biber,style=gb7714-2015]{biblatex}
% 文献表字体
% \renewcommand{\bibfont}{\zihao{-5}}
% 文献表条目间的间距
\setlength{\bibitemsep}{0pt}
% 导入参考文献数据库
\addbibresource{bibdata/thesis.bib}

% 定义图片文件目录与扩展名
\graphicspath{{figures/}}
\DeclareGraphicsExtensions{.pdf,.eps,.png,.jpg,.jpeg}

% 确定浮动对象的位置,可以使用 [H],强制将浮动对象放到这里(可能效果很差)
% \usepackage{float}

% 固定宽度的表格
% \usepackage{tabularx}

% 表格中支持跨行
\usepackage{multirow}

% 表格中数字按小数点对齐
\usepackage{dcolumn}
\newcolumntype{d}[1]{D{.}{.}{#1}}

% 使用长表格
\usepackage{longtable}

% 附带脚注的表格
\usepackage{threeparttable}

% 附带脚注的长表格
\usepackage{threeparttablex}

% 使用三线表:toprule,midrule,bottomrule。
\usepackage{booktabs}

% 算法环境宏包
\usepackage[ruled,vlined,linesnumbered]{algorithm2e}
% \usepackage{algorithm, algorithmicx, algpseudocode}

% 代码环境宏包
\usepackage{listings}
\lstnewenvironment{codeblock}[1][]%
  {\lstset{style=lstStyleCode,#1}}{}

% 物理科学和技术中使用的数学符号,定义了 \qty 命令,与 siunitx 3.0 有冲突
% \usepackage{physics}

% 直立体数学符号
\newcommand{\dd}{\mathop{}\!\mathrm{d}}
\newcommand{\ee}{\mathrm{e}}
\newcommand{\ii}{\mathrm{i}}
\newcommand{\jj}{\mathrm{j}}

% 国际单位制宏包
\usepackage{siunitx}[=v2]

% 定理环境宏包
\usepackage{ntheorem}
% \usepackage{amsthm}

% 绘图宏包
\usepackage{tikz}
\usetikzlibrary{shapes.geometric, arrows}

% 一些文档中用到的 logo
\usepackage{hologo}
\newcommand{\XeTeX}{\hologo{XeTeX}}
\newcommand{\BibLaTeX}{\textsc{Bib}\LaTeX}

% 借用 ltxdoc 里面的几个命令方便写文档
\DeclareRobustCommand\cs[1]{\texttt{\char`\\#1}}
\providecommand\pkg[1]{{\sffamily#1}}

% 自定义命令

% E-mail
\newcommand{\email}[1]{\href{mailto:#1}{\texttt{#1}}}

% hyperref 宏包在最后调用
\usepackage{hyperref}

\begin{document}

%TC:ignore

% 标题页
\maketitle

% 原创性声明及使用授权书
% \copyrightpage
% 插入外置原创性声明及使用授权书
% \copyrightpage[scans/sample-copyright-old.pdf]

% 前置部分
\frontmatter

% 摘要
% !TEX root = ../main.tex

\begin{abstract*}
  In this project, we first implement a Memory Access Tracing (MAT) method to trace the number of times a given range of memory is written by a particular task. Then, based on the Memory Access Tracing, we design a Race-Averse Linux Process Scheduler (RAS), using a Weighted Round Robin style according to race probabilities of each task. Furthermore, we benchmark the performance of Linux native schedulers and our Race-Averse Scheduler. The results prove that RAS achieves higher throughput, shorter turnaround time and lower latency.
\end{abstract*}


% 目录
\tableofcontents
% 插图索引
\listoffigures*
% 表格索引
\listoftables*
% 算法索引
\listofalgorithms*

% 符号对照表
% !TEX root = ../main.tex

\begin{nomenclature*}
\label{chap:symb}

\begin{longtable}{rl}
  $\epsilon$  & 介电常数  \\  
  $\mu$       & 磁导率    \\
  $\epsilon$  & 介电常数  \\
  $\mu$       & 磁导率    \\
  $\epsilon$  & 介电常数  \\
  $\mu$       & 磁导率    \\
  $\epsilon$  & 介电常数  \\
  $\mu$       & 磁导率    \\
  $\epsilon$  & 介电常数  \\
  $\mu$       & 磁导率    \\
  $\epsilon$  & 介电常数  \\
  $\mu$       & 磁导率    \\
  $\epsilon$  & 介电常数  \\
  $\mu$       & 磁导率    \\
  $\epsilon$  & 介电常数  \\
  $\mu$       & 磁导率    \\
  $\epsilon$  & 介电常数  \\
  $\mu$       & 磁导率    \\
  $\epsilon$  & 介电常数  \\
  $\mu$       & 磁导率    \\
  $\epsilon$  & 介电常数  \\
  $\mu$       & 磁导率    \\
  $\epsilon$  & 介电常数  \\
  $\mu$       & 磁导率    \\
  $\epsilon$  & 介电常数  \\
  $\mu$       & 磁导率    \\
  $\epsilon$  & 介电常数  \\
  $\mu$       & 磁导率    \\
  $\epsilon$  & 介电常数  \\
  $\mu$       & 磁导率    \\
  $\epsilon$  & 介电常数  \\
  $\mu$       & 磁导率    \\
  $\epsilon$  & 介电常数  \\
  $\mu$       & 磁导率    \\
  $\epsilon$  & 介电常数  \\
  $\mu$       & 磁导率    \\
  $\epsilon$  & 介电常数  \\
  $\mu$       & 磁导率    \\
  $\epsilon$  & 介电常数  \\
  $\mu$       & 磁导率    \\
  $\epsilon$  & 介电常数  \\
  $\mu$       & 磁导率    \\
  $\epsilon$  & 介电常数  \\
  $\mu$       & 磁导率    \\
  $\epsilon$  & 介电常数  \\
  $\mu$       & 磁导率    \\
  $\epsilon$  & 介电常数  \\
  $\mu$       & 磁导率    \\
  $\epsilon$  & 介电常数  \\
  $\mu$       & 磁导率    \\
  $\epsilon$  & 介电常数  \\
  $\mu$       & 磁导率    \\
  $\epsilon$  & 介电常数  \\
  $\mu$       & 磁导率    \\
\end{longtable}

\end{nomenclature*}


%TC:endignore

% 主体部分
\mainmatter

% 正文内容
\input{contents/intro}
\input{contents/math_and_citations}
\input{contents/floats}
% !TEX root = ../main.tex

\begin{summary}
In this project, we first implemented the Memory Accessing Tracing method. Then, based on the tracer, we designed and implemented a new Linux scheduling algorithm RAS. Next, we successfully embedded it into Linux kernel, and proved that it can work correctly. Finally, we wrote several benchmark programs to test the performance of RAS and compared RAS with Linux native scheduling algorithms, NORMAL, FIFO and RR. The benchmark results shown that RAS has higher throughput, shorter \& more stable turnaround time and slightly lower latency.

Although the results are exciting, we still have to note that RAS has many limitations. For example, Memory Access Tracing can not be done completely in the kernel so far. Therefore, if we want to use RAS, we have to manually set protection for the specified memory again and again. Besides, RAS is only sensitive to write operation. So for CPU-bound tasks, RAS can not gain a better performance. We still need to make further improvements for RAS.

For me, this is my first time to read the Linux source code. That's a really challenge. But I learned a lot from it. This project has laid a foundation for me to explore the field of system in the future.
\end{summary}


%TC:ignore

% 参考文献
\printbibliography[heading=bibintoc]

% 附录
\appendix

% 附录中图表不加入索引
\captionsetup{list=no}

% 附录内容
% !TEX root = ../main.tex

\chapter{Maxwell Equations}

选择二维情况,有如下的偏振矢量:
\begin{subequations}
  \begin{align}
    {\bf E} &= E_z(r, \theta) \hat{\bf z}, \\
    {\bf H} &= H_r(r, \theta) \hat{\bf r} + H_\theta(r, \theta) \hat{\bm\theta}.
  \end{align}
\end{subequations}
对上式求旋度:
\begin{subequations}
  \begin{align}
    \nabla \times {\bf E} &= \frac{1}{r} \frac{\partial E_z}{\partial\theta}
      \hat{\bf r} - \frac{\partial E_z}{\partial r} \hat{\bm\theta}, \\
    \nabla \times {\bf H} &= \left[\frac{1}{r} \frac{\partial}{\partial r}
      (r H_\theta) - \frac{1}{r} \frac{\partial H_r}{\partial\theta} \right]
      \hat{\bf z}.
  \end{align}
\end{subequations}
因为在柱坐标系下,$\overline{\overline\mu}$ 是对角的,所以 Maxwell 方程组中电场
$\bf E$ 的旋度:
\begin{subequations}
  \begin{align}
    & \nabla \times {\bf E} = \ii \omega {\bf B}, \\
    & \frac{1}{r} \frac{\partial E_z}{\partial\theta} \hat{\bf r} -
      \frac{\partial E_z}{\partial r}\hat{\bm\theta} = \ii \omega \mu_r H_r
      \hat{\bf r} + \ii \omega \mu_\theta H_\theta \hat{\bm\theta}.
  \end{align}
\end{subequations}
所以 $\bf H$ 的各个分量可以写为:
\begin{subequations}
  \begin{align}
    H_r &= \frac{1}{\ii \omega \mu_r} \frac{1}{r}
      \frac{\partial E_z}{\partial\theta}, \\
    H_\theta &= -\frac{1}{\ii \omega \mu_\theta}
      \frac{\partial E_z}{\partial r}.
  \end{align}
\end{subequations}
同样地,在柱坐标系下,$\overline{\overline\epsilon}$ 是对角的,所以 Maxwell 方程
组中磁场 $\bf H$ 的旋度:
\begin{subequations}
  \begin{align}
    & \nabla \times {\bf H} = -\ii \omega {\bf D}, \\
    & \left[\frac{1}{r} \frac{\partial}{\partial r}(r H_\theta) - \frac{1}{r}
      \frac{\partial H_r}{\partial\theta} \right] \hat{\bf z} = -\ii \omega
      {\overline{\overline\epsilon}} {\bf E} = -\ii \omega \epsilon_z E_z
      \hat{\bf z}, \\
    & \frac{1}{r} \frac{\partial}{\partial r}(r H_\theta) - \frac{1}{r}
      \frac{\partial H_r}{\partial\theta} = -\ii \omega \epsilon_z E_z.
  \end{align}
\end{subequations}
由此我们可以得到关于 $E_z$ 的波函数方程:
\begin{equation}
  \frac{1}{\mu_\theta \epsilon_z} \frac{1}{r} \frac{\partial}{\partial r}
  \left(r \frac{\partial E_z}{\partial r} \right) + \frac{1}{\mu_r \epsilon_z}
  \frac{1}{r^2} \frac{\partial^2E_z}{\partial\theta^2} +\omega^2 E_z = 0.
\end{equation}

% !TEX root = ../main.tex

\chapter{绘制流程图}

图~\ref{fig:flow_chart} 是一张流程图示意。使用 \pkg{tikz} 环境,搭配四种预定义节
点(\verb+startstop+、\verb+process+、\verb+decision+和\verb+io+),可以容易地绘
制出流程图。

\begin{figure}[!htp]
  \centering
  \resizebox{6cm}{!}{
% 定义流程图节点
\tikzstyle{startstop} = [
  rectangle,
  rounded corners,
  minimum width=2cm,
  minimum height=1cm,
  text centered,
  draw=black
]
\tikzstyle{io} = [
  trapezium,
  trapezium left angle=75,
  trapezium right angle=105,
  minimum width=1cm,
  minimum height=1cm,
  text centered,
  draw=black
]
\tikzstyle{process} = [
  rectangle,
  minimum width=2cm,
  minimum height=1cm,
  text centered,
  draw=black
]
\tikzstyle{decision} = [
  diamond,
  minimum width=2cm,
  minimum height=1cm,
  text centered,
  draw=black]
\tikzstyle{arrow} = [thick, ->, >=stealth]

\begin{tikzpicture}[node distance=2cm]
  % 设置节点
  \node (pic) [startstop] {待测图片};
  \node (bg) [io, below of=pic] {读取背景};
  \node (pair) [process, below of=bg] {匹配特征点对};
  \node (threshold) [decision, below of=pair, yshift=-0.5cm] {多于阈值};
  \node (clear) [decision, right of=threshold, xshift=3cm] {清晰?};
  \node (capture) [process, right of=pair, xshift=3cm, yshift=0.5cm] {重采};
  \node (matrix_p) [process, below of=threshold, yshift=-0.8cm] {透视变换矩阵};
  \node (matrix_a) [process, right of=matrix_p, xshift=3cm] {仿射变换矩阵};
  \node (reg) [process, below of=matrix_p] {图像修正};
  \node (return) [startstop, below of=reg] {配准结果};
    
  % 连接节点
  \draw [arrow](pic) -- (bg);
  \draw [arrow](bg) -- (pair);
  \draw [arrow](pair) -- (threshold);

  \draw [arrow](threshold) -- node[anchor=south] {否} (clear);

  \draw [arrow](clear) -- node[anchor=west] {否} (capture);
  \draw [arrow](capture) |- (pic);
  \draw [arrow](clear) -- node[anchor=west] {是} (matrix_a);
  \draw [arrow](matrix_a) |- (reg);

  \draw [arrow](threshold) -- node[anchor=east] {是} (matrix_p);
  \draw [arrow](matrix_p) -- (reg);
  \draw [arrow](reg) -- (return);
\end{tikzpicture}
}
  \bicaption{绘制流程图效果}{Flow chart}
  \label{fig:flow_chart}
\end{figure}


% 结尾部分
\backmatter

% 用于盲审的论文需隐去致谢、发表论文、科研成果、简历

% 致谢
% !TEX root = ../main.tex

\begin{acknowledgements}
  Thanks Prof. Wu. Your excellent teaching helps me a lot.
  
  Thanks TAs for this project. I appreciate your quick replies.
  
  Thanks my classmates. Our discussion greatly inspired me.
  
  Thanks \href{https://github.com/sjtug}{@sjtug}. Your \LaTeX\ templates have saved me a lot of time.
\end{acknowledgements}


% 发表论文、科研成果
% 盲审论文中,发表论文及科研成果等仅以第几作者注明即可,不要出现作者或他人姓名
% !TEX root = ../main.tex

\begin{achievements}

\subsection*{学术论文}

\begin{bibliolist}{00}
  \item Chen H, Chan C~T. Acoustic cloaking in three dimensions using acoustic metamaterials[J]. Applied Physics Letters, 2007, 91:183518.
  \item Chen H, Wu B~I, Zhang B, et al. Electromagnetic Wave Interactions with a Metamaterial Cloak[J]. Physical Review Letters, 2007, 99(6):63903.
\end{bibliolist}

\begin{bibliolist*}{00}
  \item 第一作者. 中文核心期刊论文, 2007.
  \item 第一作者. EI 国际会议论文, 2006.
\end{bibliolist*}

\subsection*{专利}

\begin{bibliolist}{00}
  \item 第一发明人,“永动机”,专利申请号202510149890.0
\end{bibliolist}

\begin{bibliolist*}{00}
  \item 第一发明人,“永动机”,专利申请号XXXXXXXXXXXX.X
\end{bibliolist*}

\end{achievements}


% 简历
% !TEX root = ../main.tex

\begin{resume}
  \subsection*{基本情况}
    某某,yyyy 年 mm 月生于 xxxx。

  \subsection*{教育背景}
  \begin{itemize}
    \item yyyy 年 mm 月至今,上海交通大学,博士研究生,xx 专业
    \item yyyy 年 mm 月至 yyyy 年 mm 月,上海交通大学,硕士研究生,xx 专业
    \item yyyy 年 mm 月至 yyyy 年 mm 月,上海交通大学,本科,xx 专业
  \end{itemize}

  \subsection*{研究兴趣}
    \LaTeX{} 排版

  \subsection*{联系方式}
  \begin{itemize}
    \item 地址: 上海市闵行区东川路 800 号,200240
    \item E-mail: \email{xxx@sjtu.edu.cn}
  \end{itemize}
\end{resume}


% 学士学位论文要求在最后有一个大摘要,单独编页码
% !TEX root = ../main.tex

\begin{digest}
  An imperial edict issued in 1896 by Emperor Guangxu, established Nanyang
  Public School in Shanghai. The normal school, school of foreign studies,
  middle school and a high school were established. Sheng Xuanhuai, the person
  responsible for proposing the idea to the emperor, became the first president
  and is regarded as the founder of the university.

  During the 1930s, the university gained a reputation of nurturing top
  engineers. After the foundation of People's Republic, some faculties were
  transferred to other universities. A significant amount of its faculty were
  sent in 1956, by the national government, to Xi'an to help build up Xi'an Jiao
  Tong University in western China. Afterwards, the school was officially
  renamed Shanghai Jiao Tong University.

  Since the reform and opening up policy in China, SJTU has taken the lead in
  management reform of institutions for higher education, regaining its vigor
  and vitality with an unprecedented momentum of growth. SJTU includes five
  beautiful campuses, Xuhui, Minhang, Luwan Qibao, and Fahua, taking up an area
  of about 3,225,833 m2. A number of disciplines have been advancing towards the
  top echelon internationally, and a batch of burgeoning branches of learning
  have taken an important position domestically.

  Today SJTU has 31 schools (departments), 63 undergraduate programs, 250
  masters-degree programs, 203 Ph.D. programs, 28 post-doctorate programs, and
  11 state key laboratories and national engineering research centers.

  SJTU boasts a large number of famous scientists and professors, including 35
  academics of the Academy of Sciences and Academy of Engineering, 95 accredited
  professors and chair professors of the "Cheung Kong Scholars Program" and more
  than 2,000 professors and associate professors.

  Its total enrollment of students amounts to 35,929, of which 1,564 are
  international students. There are 16,802 undergraduates, and 17,563 masters
  and Ph.D. candidates. After more than a century of operation, Jiao Tong
  University has inherited the old tradition of "high starting points, solid
  foundation, strict requirements and extensive practice." Students from SJTU
  have won top prizes in various competitions, including ACM International
  Collegiate Programming Contest, International Mathematical Contest in Modeling
  and Electronics Design Contests. Famous alumni include Jiang Zemin, Lu Dingyi,
  Ding Guangen, Wang Daohan, Qian Xuesen, Wu Wenjun, Zou Taofen, Mao Yisheng,
  Cai Er, Huang Yanpei, Shao Lizi, Wang An and many more. More than 200 of the
  academics of the Chinese Academy of Sciences and Chinese Academy of
  Engineering are alumni of Jiao Tong University.
\end{digest}


%TC:endignore

\end{document}
